% This is samplepaper.tex, a sample chapter demonstrating the
% LLNCS macro package for Springer Computer Science proceedings;
% Version 2.21 of 2022/01/12
%
\documentclass[runningheads]{llncs}

\pagestyle{plain}
\usepackage{lineno}
\linenumbers 

\usepackage[T1]{fontenc}
% T1 fonts will be used to generate the final print and online PDFs,
% so please use T1 fonts in your manuscript whenever possible.
% Other font encondings may result in incorrect characters.
%
\usepackage{graphicx}
% Used for displaying a sample figure. If possible, figure files should
% be included in EPS format.
%
% If you use the hyperref package, please uncomment the following two lines
% to display URLs in blue roman font according to Springer's eBook style:
%\usepackage{color}
%\renewcommand\UrlFont{\color{blue}\rmfamily}
%
\begin{document}
%
\title{Building a Query Language?\thanks{Supported by organization x.}}
%
%\titlerunning{Abbreviated paper title}
% If the paper title is too long for the running head, you can set
% an abbreviated paper title here
%
\author{First Author\inst{1}\orcidID{0000-1111-2222-3333} \and
Second Author\inst{2,3}\orcidID{1111-2222-3333-4444} \and
Third Author\inst{3}\orcidID{2222--3333-4444-5555}}
%
\authorrunning{F. Author et al.}
% First names are abbreviated in the running head.
% If there are more than two authors, 'et al.' is used.
%
\institute{Princeton University, Princeton NJ 08544, USA \and
Springer Heidelberg, Tiergartenstr. 17, 69121 Heidelberg, Germany
\email{lncs@springer.com}\\
\url{http://www.springer.com/gp/computer-science/lncs} \and
ABC Institute, Rupert-Karls-University Heidelberg, Heidelberg, Germany\\
\email{\{abc,lncs\}@uni-heidelberg.de}}
%
\maketitle              % typeset the header of the contribution
%
\begin{abstract}
The abstract should briefly summarize the contents of the paper in
150--250 words.

\keywords{First keyword  \and Second keyword \and Another keyword.}
\end{abstract}
%
%
%
\section{Introduction}

TODO: need a query that shows the expressiveness or power of the query language, comparing it with the same query 
written in other query languages

What should be the story/pitch here?

- semi-structured data is prevalent in many places, including modern web applications.

- languages like graphql, JQ, JSONiq, SQL++, etc. exists to handle such data.

- however, they are less expressive, doesn't permit (easy) code gen, cannot operate inside the browser, etc.

- in this work, we present a new query language, that is targeted to query nested structures and produce
nested structures as results, resembles existing object notation (like GraphQL), permit query optimization (e.g., de-correlation)
and efficient code generation, and compositional and easy to meta-program


RPAI~\cite{rpai} (need at least one cite for bibtex :-P)

The query language provides a unified language that handles multiple aspects:

\begin{enumerate}
    \item High-performance nested, hetrogenius querying language (like JSONiq or JQ, XQuery)
    that is compiled for maximal performance

    \item permit query optimization and code generation, utilizing an existing compiler
    infrastructure

    \item can run in the browser (everything in JS), essentially making every 

    \item an expressive language that provides a unified representation of functions,
     records, and tables the provides an easier way to specify front-end (dashboard-style) logic
    (think GraphQL, but more powerful with having GUI elements, etc. part of the query?)
    \item more capabilities like recursive queries (with semi-naive evaluation),
    incrementality, etc.
\end{enumerate}

\section{Background/Motivation}

\section{Design}

\section{Experiments}

\section{Related Work}

\section{Future Work}
%
% ---- Bibliography ----
%
% BibTeX users should specify bibliography style 'splncs04'.
% References will then be sorted and formatted in the correct style.

\bibliographystyle{splncs04}
\bibliography{references}
\end{document}
