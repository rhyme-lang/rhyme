% This is samplepaper.tex, a sample chapter demonstrating the
% LLNCS macro package for Springer Computer Science proceedings;
% Version 2.21 of 2022/01/12
%
\documentclass[runningheads]{llncs}

\pagestyle{plain}
\usepackage{lineno}
\linenumbers 

\newcommand{\lang}{QL-TBD}

\usepackage[T1]{fontenc}
% T1 fonts will be used to generate the final print and online PDFs,
% so please use T1 fonts in your manuscript whenever possible.
% Other font encondings may result in incorrect characters.
%
\usepackage{graphicx}
% Used for displaying a sample figure. If possible, figure files should
% be included in EPS format.
%
% If you use the hyperref package, please uncomment the following two lines
% to display URLs in blue roman font according to Springer's eBook style:
%\usepackage{color}
%\renewcommand\UrlFont{\color{blue}\rmfamily}
%
\begin{document}
%
\title{Building a Query Language?}

%
%\titlerunning{Abbreviated paper title}
% If the paper title is too long for the running head, you can set
% an abbreviated paper title here
%
% \author{Supun Abeysinghe\inst{1}\orcidID{0000-1111-2222-3333} \and
% Tiark Rompf\inst{2,3}\orcidID{1111-2222-3333-4444}}
% \author{Supun Abeysinghe \and Tiark Rompf}
% %
% \authorrunning{F. Author et al.}
% % First names are abbreviated in the running head.
% % If there are more than two authors, 'et al.' is used.
% %
% \institute{Purdue University, West Lafayette IN 47906, USA\\
% \email{\{tabeysin,tiark\}@purdue.com}
% }
%
\maketitle              % typeset the header of the contribution
%
\begin{abstract}
We propose an expressive language for high-level data manipulation that
mainly targets querying nested structures (e.g., JSON) and produce nested
structures as results.
Our query language resembles existing object notation, compositional, and
permits query optimization and code generation via construction of an
intermediate representation (IR).


\keywords{First keyword  \and Second keyword \and Another keyword.}
\end{abstract}
%
%
%
\section{Introduction}

Modern applications are typically implemented in a tiered manner, with each tier implemented
using different kind of techniques.
For example, the front-end handles the visual aspects of the application and is typically
written using a mix of HTML, CSS, and JavaScript (specifically, frameworks like
React, Angular, etc.). 
Front-end communicates with a middle-tier, often referred to as the
application tier via techniques like REST end points or more general query languages like
GraphQL to query the data that needs to be presented to the end user.
The application layer then issues queries to a third, database tier that runs the issued
queries on the stored data to produce the desired output.

Most of these tiers manipulate data in semi-structured formats like JSON
and there are a variety of query languages that is tailored to query data in these kinds of
formats.
Examples include, JSONiq, AQL, SQL++, UNQL, JAQL, etc.
While all these query languages provide expressive languages to query these
types of data, these languages tend to be targeted towards large-scale
analytics engines (e.g., Spark-like workloads) and not much emphasis is put
on the front-end logic.
In contrast, languages like GraphQL has gained popularity since it provides
a simple and intuitive interface to query data from backends.
But the actual logic for evaluating these queries (called resolvers) should be
written manually in most cases and involves invoking queries to a database.


In this work, we present a new query language called \lang, an expressive language
targeted towards high-level data manipulation.
\lang is used to query nested structures (like JSON), produce nested structures as 
result with a query syntax that resembles existing object notation.
\lang is designed to be compositional and, in a way that it permits query
optimization and code generation.



- our direction: 

-- since 


Generally, building the front end is centered around the 


Consider a dashboard that displays stocks of a warehouse: ....


- various ways of data processing exists

- data tends to be semi-structured in many settings (web applications: json)

- querying this data is really important (say building a dashboard)

- there are existing languages that targets these types of workloads

- on one end there are full DB-style commercial offerings like CouchBase, MongoDB

-- they are focused on scale and multi-node large scale data processing, transactions, etc. 
    and not generally used in frontend querying

- then there are languages like GraphQL that enables querying data from the frontend. 
   These are arguably more user-friendly  

- we target this scenarios like this

- Many GraphQL implementations rely on writing resolvers to handle the queries. These
  resolvers usually call backend databases to retrieve data and do different kinds of processing
  (e.g., filtering, aggregates, etc.) manually to compute the final result.


- lightweight and intuitive language, but sufficiently expressive to cover a wide range 
   of practical JSON queries

- language not only contains JSON analytics, but also GUI components

- with idea, your program becomes a query


Relational data processing is most widely used. But only on structured data.
But these days there is a tendency to store semi-structured data.
This give emergnce to systems that can handle semi-structured
 - commercial: data lakes, no-sql like crunchbase (SQL++), MongoDB, etc.
Some systems use query languages that mirrors SQL syntax to make it appealing/easier
to adapt for developers who know SQL.
Other query languages also exists like JSONiq (inspired by XQuery), JQ, ...

In this work, we propose a new query language, specifically targeting semi-structured
data, but is flexible to perform other types of operations as well.
We take inspiration from ....



Various query languages exists ranging from SQL, GraphQL, 

TODO: need a query that shows the expressiveness or power of the query language, comparing it with the same query 
written in other query languages

What should be the story/pitch here?

- semi-structured data is prevalent in many places, including modern web applications.

- languages like graphql, JQ, JSONiq, SQL++, etc. exists to handle such data.

- however, they are less expressive, doesn't permit (easy) code gen, cannot operate inside the browser, etc.

- in this work, we present a new query language, that is targeted to query nested structures and produce
nested structures as results, resembles existing object notation (like GraphQL), permit query optimization (e.g., de-correlation)
and efficient code generation, and compositional and easy to meta-program


RPAI~\cite{rpai} (need at least one cite for bibtex :-P)

The query language provides a unified language that handles multiple aspects:

\begin{enumerate}
    \item High-performance nested, hetrogenius querying language (like JSONiq or JQ, XQuery)
    that is compiled for maximal performance

    \item permit query optimization and code generation, utilizing an existing compiler
    infrastructure

    \item can run in the browser (everything in JS), essentially making every 

    \item an expressive language that provides a unified representation of functions,
     records, and tables the provides an easier way to specify front-end (dashboard-style) logic
    (think GraphQL, but more powerful with having GUI elements, etc. part of the query?)
    \item more capabilities like recursive queries (with semi-naive evaluation),
    incrementality, etc.
\end{enumerate}

\section{The Query Language}
\subsection{Basics}
aggregation, etc.

\subsection{Group By}
\subsection{Filters}
\subsection{Join}
\subsection{Used-defined Functions}
\subsection{Arrays}
\subsection{Fluent API}

\subsection{GUI Components}
Basic building blocks -- table definition, and generic display, svg, etc. 

\subsection{Example: Pivot Table}
How to use group bys, partial aggregates, display, svg, etc.

\section{The IR}
\subsection{IR Structure}
Loop-free and branch-free code, program structure implicit in dependencies

\section{Code Generation}
Scheduling based on dependencies, lambda lifting to de-correlate queries

\section{Implementation}
We implement this in JavaScript so that it can run in the browser.

\section{Experiments}

\section{Related Work}

\section{Future Work}
Incrementality, Datalog-style recursive queries?, partial re-evaluation as
query changes (with minimal recomputation)
%
% ---- Bibliography ----
%
% BibTeX users should specify bibliography style 'splncs04'.
% References will then be sorted and formatted in the correct style.

\bibliographystyle{splncs04}
\bibliography{references}
\end{document}
